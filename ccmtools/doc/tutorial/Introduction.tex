% $Id: Introduction.tex,v 1.9 2005/03/17 08:18:03 teiniker Exp $

%==============================================================================
\chapter{CCM Tools Overview}
%==============================================================================

The CCM Tools are a set of Java programs and libraries as well as Python scripts
that supports component development, based on the {\it CORBA Component
Model} (CCM) \cite{CCMSpecification} and our local component extensions. 

\begin{figure}[htbp]
    \begin{center}
        \includegraphics [width=12cm,angle=0] {ComponentGeneratorTools}
        \caption{CCM Tools framework}
        \label{ccmtools}
    \end{center}
\end{figure}

As shown in Fig.~\ref{ccmtools}, the CCM Tools form a component development 
framework that is flexible and extensible.


\newpage
Using a well defined {\bf CCM model}, we can separate component descriptions 
from code generator tools. Therefore, we can add new description methods (e.g.
UML profiles) or code generator tools (e.g. a Java Component Generator) without 
changing other tools.

The current CCM Tools framework contains the following tools:
\begin{description}
\item [uml2idl Generator]
Using a UML profile for CCM \cite{} we can model component definitions with
UML tools. From a given UML XMI file a uml2idl generator extracts a IDL3 
and an OCL file. 
While the IDL3 file is used to generate component logic, the OCL file is input
for a Design by Contract (DbC) generator (as described later).

\item [CCM Metamodel Library]
The CCM specification defines an {\it Interface Repository Metamodel} 
corresponding to the IDL3 syntax. We implemented a CCM Metamodel Library that 
allows creation and manipulation of CCM models. Using this library we can 
clearly separate the parser tools and code generators. 
The parser creates a model object for every part of the
source code matching an IDL grammar rule.

\item [idl3 Parser]
The idl3 Parser reads IDL3 files, checks the syntax of these source code
and creates a CCM model using the CCM Metamodel Library in the memory. This CCM
model is the starting point for all code generator tools in the framework.

\item [idl3 Generator]
To test the functionality of the CCM Metamodel Library and the IDL3 Parser we 
implemented an IDL3 Generator that writes out a CCM model into corresponding 
IDL3 Files.

\item [idl3mirror Generator]
We use a {\it Test-Driven Development} (TDD) strategy to develop and to specify
components (as described later in this tutorial). Our component unit test uses a
mirror component to connect all ports of an existing component. The idl3mirror
Generator creates the IDL3 syntax definition of this mirror component.


\item [OCL Metamodel Library]
We have implemented an OCL metamodel library that is used by the OCL parser
to create an OCL model in memory.
This in memory OCL model can be used by generator tools to retrieve
information for code generation.

\item [OCL Parser]
From an OCL source file, an OCL parser performs some syntax checks and
establishes an OCL model in memory using the OCL metamodel library. 
This model is the starting point for code generator tools based on OCL. 

\item [c++local Generator]
The component model is realized by the component logic that implements the
operations for providing, using and connecting components by their facets and
receptacles. We implemented a generator tool that creates the local component
logic starting from a given CCM model. After generating the component logic, 
component developers can write business logic into the generated
component skeletons.

\item [c++dbc Generator]
From an OCL model and a CCM model this generator creates a set of
Design by Contract adapter for local C++ components.
These adapters can be used alternatively to the regular local component
adapters to support DbC.

\item [c++local-test Generator]
After building a local C++ component we can run a unit test procedure. To
provide a suitable test environment we generate a test client implementation
that instantiates a component as well as its mirror component and manages this 
component unit test. 

\item [idl2 Generator]
To implement CORBA components the IDL3 source code is mapped to IDL2 that can be
processed by a usual {\bf IDL2 Compiler} (currently we use Mico ORB). The
transformation from IDL3 to IDL2 extends the IDL2 source code with some operations 
needed for navigation between components and their ports (equivalent operations). 
We support this transformation with a idl2 Generator that creates IDL2 files from 
a existing CCM model.

\item [c++remote Component Generator]
The local components only works in a common address space and must be
implemented in the same programming language. To overcome these limitations we
can generate remote component logic that interfaces the local components with CORBA.
The remote component logic is thus a superset of the local component logic. Note
that the external view of such a remote component quite conforms to the CCM
specification.

\item [c++remote-test Generator]
After generation of remote components we can also run a remote test procedure.
In parallel to the local test environment, we can use the mirror component test 
concept in the remote case too.
Thus, a generated remote test client connects a remote component with its
mirror component and manages this remote unit test. 
\end{description}


%==============================================================================
\section{CCM Tools runtime architecture}
%==============================================================================

The CCM Tools source code is available from {\tt ccmtools.sourceforge.net}
and is organized in the following structure (Fig.~\ref{ccmtools-structure}):

\begin{figure}[htbp]
    \begin{center}
        \includegraphics [width=12cm,angle=0] {CcmToolsArchitecture}
        \caption{CCM Tools structure}
        \label{ccmtools-structure}
    \end{center}
\end{figure}

\begin{description}
\item [ccmtools:]
This module contains all the tools and libraries that build the CCM Tools 
framework in form of Java source code.

\item [cpp-environment:]
This module contains C++ source code that builds a runtime environment for
generated components (e.g. C++ smart pointer library, CCM header files, 
HomeFinder implementations etc.).
The C++ components depend only on cpp-environment libraries. 
Remote components are also based on {\bf Mico} \cite{MicoORB} an open source 
Object Request Broker (ORB) implementation.


\item [java-environment:]
This module contains Java code that is used by Java clients
as a runtime library to access remote CCM components.

\end{description}

\newpage
% $Id: Installation.tex,v 1.13 2004/10/12 18:16:16 teiniker Exp $
%==============================================================================
\section{CCM Tools installation}
%==============================================================================

%------------------------------------------------------------------------------
\subsection{Prerequisites}
%------------------------------------------------------------------------------

To install the CCM Tools, the following programs must be available:
\begin{description}
\item [Java SDK $\ge$ 1.4] ({\tt http://java.sun.com/j2se})
\item [Apache Ant $\ge$ 1.5.3] ({\tt http://ant.apache.org})
\item [Python $\ge$ 2.2] ({\tt http://python.org})
\item [cpp $\ge$ 2.96] ({\tt http://www.gnu.org})
\end{description}

To build the generated local C++ components, we also need:
\begin{description}
\item [Confix $\ge$ 1.3pre14] ({\tt http://confix.sourceforge.net})
\item [gcc $\ge$ 2.95.3] ({\tt http://www.gnu.org})
\end{description}

To generate and build remote C++ components, we additionally need:
\begin{description}
\item [mico $\ge$ 2.3.10] ({\tt http://www.mico.org/})
\end{description}


%------------------------------------------------------------------------------
\subsection{How to get it}
%------------------------------------------------------------------------------

The project is hosted at Sourceforge ({\tt http://ccmtools.sf.net}). See the web
site for releases and announcements.

You can also subscribe to the {\tt ccmtools-announce} mailing list for CCM Tools
release announcements. The {\tt ccmtools-users} mailing list provides a forum
for discussion about using the CCM Tools.



%------------------------------------------------------------------------------
\subsection{Source distribution}
%------------------------------------------------------------------------------
Installing of the CCM Tools framework requires the following steps once you get 
the source code. 
(Pretend that the source tarballs you got were {\tt ccmtools-A.B.X.tar.gz},
{\tt cpp-environment-A.B.X.tar.gz} and {\tt java-environment-A.B.X.tar.gz}).

\begin{small}
\begin{verbatim}
 $ tar xvzf ccmtools-A.B.X.tar.gz
 $ tar xvzf cpp-environment-A.B.X.tar.gz
 $ tar xvzf java-environment-A.B.X.tar.gz
\end{verbatim}
\end{small}

Alternatively, you can check out an up-to-date version from CVS:
\begin{small}
\begin{verbatim}
 $ cvs -d :pserver:anonymous@cvs.sf.net:/cvsroot/ccmtools login
 Password: <press enter>
 $ cvs -d :pserver:anonymous@cvs.sf.net:/cvsroot/ccmtools co ccmtools
 $ cvs -d :pserver:anonymous@cvs.sf.net:/cvsroot/ccmtools co cpp-environment
 $ cvs -d :pserver:anonymous@cvs.sf.net:/cvsroot/ccmtools co java-environment
\end{verbatim}
\end{small}

Now we should have the following directory structure: 
\begin{small}
\begin{verbatim}
  |-- ccmtools/
  |-- cpp-environment/
  `-- java-environment/
\end{verbatim}
\end{small}


%------------------------------------------------------------------------------
\subsubsection{Install java-environment package:}
%------------------------------------------------------------------------------
To build and install the java-environment we use Ant:
\begin{small}
\begin{verbatim}
 $ cd java-environment
 $ ant install -Dprefix=<install-path>
 
 # Don't forget to set the CLASSPATH environment veriable
 $ export CLASSPATH=<install-path>/lib/antlr.java:   \
                    <install-path>/lib/dtd2java.jar: \
                    <install-path>/lib/jdom.jar:     \     
                    <install-path>/lib/mdr01.jar:    \
                    <install-path>/lib/uml2idl.jar:  \
                    $CLASSPATH
\end{verbatim}
\end{small}



%------------------------------------------------------------------------------
\subsubsection{Install ccmtools package:}
%------------------------------------------------------------------------------
To build and install the  ccmtools package we also use Ant:
\begin{small}
\begin{verbatim}
 # For using local components: 
 $ cd ccmtools
 $ ant install -Dprefix=<CCM_INSTALL_PATH>

 # Don't forget to set the CLASSPATH environment veriable
 $ export CCMTOOLS_HOME=<CCM_INSTALL_PATH>
 $ export CLASSPATH=<CCM_INSTALL_PATH>/lib/ccmtools.jar:    \
                    <CCM_INSTALL_PATH>/lib/oclmetamodel.jar
 $ export PATH=$CCMTOOLS_HOME/bin:$PATH     

 # For using remote components too: 
 $ export CCM_COMPONENT_REPOSITORY=${CCMTOOLS_HOME} 
 $ export CCM_NAME_SERVICE=corbaloc:iiop:1.2@localhost:5050/NameService
\end{verbatim}
\end{small}
 
OK, that's it! \\
Now we are ready to generate CCM components using the CCM Tools framework.


%------------------------------------------------------------------------------
\subsubsection{Install cpp-environment package:}
%------------------------------------------------------------------------------
As shown in Fig.~\ref{ccmtools-structure}, to compile and run generated CCM
components, we need a C++ runtime library called cpp-environment.
To build this cpp-environment package as well as all components we will generate,
Confix is used as C++ build tool. 

First, we have to create a CCM Tools profile in Confix' configuration file
({\tt .confix}), as described in the Confix manual ;-)
\begin{small}
\begin{verbatim}
ccm_tools_profile = {
    'PREFIX': '<CCM_INSTALL_PATH>',        # use your own path!
    'BUILDROOT': '/tmp',                   # use your own path!
    'ADVANCED': 'true',
    'CONFIX': {
    },
    'CONFIGURE': {
       'ENV': {
          'CC': '/usr/local/gcc/bin/gcc',  # use your own path!
          'CXX': '/usr/local/gcc/bin/g++', # use your own path!    
          'CFLAGS': "-g -O0 -Wall",
          'CXXFLAGS': "-g -O0 -Wall",
          },
       # use your own mico install path!
       'ARGS': ['--with-external_mico=/usr/local/mico']
       },
    }

PROFILES = {
    'ccmtools': ccm_tools_profile,
    'default' : ccm_tools_profile
}
\end{verbatim}
\end{small}
It's important that you substitute your own paths in the {\tt .confix} file. \\
We can configure the {\tt ccm\_tools\_profile} as default profile, thus we don't need to use the {\tt --profile=ccmtools} confix option.
Additionally, we advise to set the {\tt ADVANCED} flag to {\tt true} instead of
using the {\tt --advanced} command--line option. 

At last, we are ready to install the cpp-environment:
\begin{small}
\begin{verbatim}
 $ cd cpp-environment

 # Install the local component's environment:
 $ confix.py --packageroot=`pwd`/WX_Utils --bootstrap --configure \
             --make --targets="all install"
 $ confix.py --packageroot=`pwd`/CCM_Local --bootstrap --configure \
             --make --targets="all install"

 # Install the remote component's environment:
 $ confix.py --packageroot=`pwd`/external --bootstrap --configure \
             --make --targets="all install"
 $ confix.py --packageroot=`pwd`/CCM_Remote --bootstrap --configure \
             --make --targets="all install"
\end{verbatim}
\end{small}

Perfect, all tools and libraries have been installed and are ready to work!






